% Interactive Complex Metacognition (ICM)
% Formal Research Paper Skeleton
% Author(s): Benjamin Sleeper
% -----------------------------------------------------------------------------
\documentclass[11pt]{article}

%------------------------------------------------------------------------------
% Packages
%------------------------------------------------------------------------------
\usepackage[utf8]{inputenc}
\usepackage[T1]{fontenc}
\usepackage{lmodern}
\usepackage{geometry}
    \geometry{margin=1in}
\usepackage{setspace}
    \onehalfspacing
\usepackage{amsmath, amssymb, amsfonts, amsthm}
\usepackage{physics}            % convenient math operators (optional)
\usepackage{graphicx}
\usepackage{hyperref}
    \hypersetup{colorlinks=true, linkcolor=blue, citecolor=blue, urlcolor=blue}
\usepackage{xcolor}
\usepackage{doi}
\usepackage{authblk}            % multiple author affiliations
\usepackage{cite}               % compressed numerical citations
\usepackage{enumitem}           % compact itemize/enumerate
\usepackage{caption}
\usepackage{subcaption}

%------------------------------------------------------------------------------
% Custom commands & math macros
%------------------------------------------------------------------------------
\newcommand{\bx}{\mathbf{x}}
\newcommand{\bu}{\mathbf{u}}
\newcommand{\bp}{\mathbf{p}}
\newcommand{\R}{\mathbb{R}}
\newcommand{\Win}{W_{\text{in}}}
\newcommand{\Wout}{W_{\text{out}}}

% Uncomment the line below to remove equation numbers on un–referenced equations
% \mathtoolsset{showonlyrefs}

%------------------------------------------------------------------------------
% Title, authors, affiliations
%------------------------------------------------------------------------------
\title{Interactive Complex Metacognition:\\A Reservoir–Based Interface for Human--AI Telepathy}

\author[1]{Benjamin Sleeper}
\author[2]{Claude 3 Opus}
\author[3]{ChatGPT}
\affil[1]{Department of Human Centered Computing, Clemson University\\\texttt{brsleep@clemson.edu}}

\date{\today}

%------------------------------------------------------------------------------
\begin{document}
\maketitle

%------------------------------------------------------------------------------
\begin{abstract}
% TODO: Summarize purpose, methods, results, significance (\textasciitilde150 words).
\end{abstract}

%------------------------------------------------------------------------------
\section{Introduction}\label{sec:intro}

Human--AI collaboration today often relies on text, graphs, or opaque dashboards to convey an AI's internal state—yet these channels miss the richness of complex dynamical patterns. We propose a paradigm shift: treat the \\emph{interface itself} as the computational substrate, allowing two ``minds'' (human~$\leftrightarrow$~AI, or human~$\leftrightarrow$~human) to exchange information by steering a shared dynamical system. By exposing the reservoir's state directly to users, \emph{Interactive Complex Metacognition} (ICM) leverages \emph{generalized synchronization} to create a meta‑communication channel through emergent patterns.

% ------------------ Contributions (edit as needed) ---------------------------
\paragraph{Contributions.} This paper:
\begin{enumerate}[topsep=2pt,itemsep=2pt,parsep=0pt]
    \item formalizes ICM as a bidirectional communication framework grounded in reservoir computing and synchronization theory;
    \item derives theoretical guarantees for information fidelity under generalized synchronization;
    \item presents a reference JavaScript implementation with real‑time particle dynamics;
    \item demonstrates two case studies: chaotic signal forecasting and human–AI intent sharing;
    \item discusses limitations, ethical considerations, and future research directions.
\end{enumerate}

%------------------------------------------------------------------------------
\section{Background}\label{sec:background}
\subsection{Reservoir Computing}
% TODO: Brief overview and key equations.

\subsection{Generalized Synchronization}
% TODO: Define GS, cite relevant literature.

%------------------------------------------------------------------------------
\section{Interactive Complex Metacognition Framework}\label{sec:framework}

\subsection{Dynamical Core}
Consider a high‑dimensional state $\bx(t)\in\R^{N}$ governed by the nonlinear update
\begin{equation}\label{eq:dynamical-core}
    \bx(t+1) = f\bigl(\Win\,\bp(t) + W\,\bx(t)\bigr),
\end{equation}
where $\bp(t)\in\R^{K}$ is a real‑time \emph{intent vector} manipulated via sliders, gestures, or programmatic control, $\Win$ maps intents into the reservoir, and $W$ denotes recurrent weights. A JavaScript‑like pseudocode snippet for a particle reservoir is given in Appendix~\ref{app:pseudocode}.

\subsection{Generalized Readout and Synchronization}
Under generalized synchronization, there exists a smooth map $\mathbf{h}$ such that the decoded output
\begin{equation}
    \hat y(t) = \mathbf{h}\bigl(\bx(t)\bigr)
\end{equation}
tracks the target signal with bounded error. Exposing $\bx(t)$ directly allows an observing mind to \emph{perceive} and internally approximate $\mathbf{h}$ without a hidden linear readout.

\subsection{Cognitive Interaction Loop}
The bidirectional loop is
\begin{equation*}
    \text{Encode}_{\text{Mind A}}\;\rightarrow\;\text{Evolve}_{\text{Reservoir}}\;\rightarrow\;\text{Perceive}_{\text{Mind B}}\;\rightarrow\;\text{Decode/Encode}_{\text{Mind B}}\;\rightarrow\;\cdots
\end{equation*}
which iterates until a shared mental model emerges.

\subsection{Compute–Optional, Compute–Enhanced Modes}
Even without heavy compute, the raw UI conveys rich patterns. Additional modules—e.g., adaptive kernels or predictive seeding—can be attached to refine expressivity while remaining grounded in synchronization theory.

%------------------------------------------------------------------------------
\section{Implementation}\label{sec:implementation}
% TODO: Software stack, system diagram, algorithmic details.

%------------------------------------------------------------------------------
\section{Experiments and Case Studies}\label{sec:experiments}
\subsection{Chaotic Forecasting}
% TODO: Experimental setup and quantitative results.

\subsection{Human–AI Intent Sharing}
% TODO: Qualitative analysis, user study protocol.

%------------------------------------------------------------------------------
\section{Discussion}\label{sec:discussion}
% TODO: Interpret results, limitations, ethical considerations.

%------------------------------------------------------------------------------
\section{Future Work}\label{sec:future-work}
% TODO: Roadmap and open questions.

%------------------------------------------------------------------------------
\section{Conclusion}\label{sec:conclusion}
% TODO: Recap contributions and significance.

%------------------------------------------------------------------------------
\section*{Acknowledgments}
% TODO: Funding sources, thanks to collaborators, etc.

%------------------------------------------------------------------------------
% Bibliography
%------------------------------------------------------------------------------
\bibliographystyle{unsrt}
\bibliography{references}

%------------------------------------------------------------------------------
% Appendices
%------------------------------------------------------------------------------
\appendix

\section{JavaScript Reservoir Pseudocode}\label{app:pseudocode}
\begin{verbatim}
// reservoir particle update (excerpt)
this.x      = W/2 + radius * Math.cos(angle);
this.y      = H/2 + radius * Math.sin(angle);
this.speedX = Math.random() * (2 * spinSpeed) - spinSpeed;
this.speedY = Math.random() * (2 * spinSpeed) - spinSpeed;
\end{verbatim}

\end{document}
